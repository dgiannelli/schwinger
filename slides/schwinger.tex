\part{Modello di Schwinger}
\makepart

\begin{frame}
    \frametitle{Un modello giocattolo}
    \begin{itemize}
        \item Idee e algoritmi possono essere testati in un modello semplificato
        \item Il modello di Schwinger è una teoria $U(1)$ in due dimensioni
        \item Condivide diverse proprietà con la QCD
        \item La principale è il confinamento dei fermioni
        \item $F_{ij}$ è ora il tensore elettromagnetico in 2D
        \item La carica topologica si riduce a:
            $$Q = \frac{g}{4\pi}\int\mathrm d^2x\,\epsilon_{ij}F_{ij}$$
    \end{itemize}
\end{frame}

\newcommand\plaquette{\raisebox{-0.08em}\Box}

\begin{frame}
    \frametitle{Sul reticolo}
    \begin{itemize}
        \item $F_{ij}$ è contenuto nelle plaquette $\plaquette$:
            $$\plaquette \simeq e^{iea^2F_{ij}}$$
        \item La carica topologica è approssimata da:
            $$Q = \frac{g}{4\pi}\int\mathrm d^2x\,\epsilon_{ij}F_{ij} \simeq %
            \frac{1}{2\pi}\sum\nolimits_\Box\Im\log\plaquette = %
            \frac{1}{2\pi}\sum\nolimits_\Box\arg\plaquette$$
        \item Gli elementi di $U(1)$ sono rappresentati da $e^{i\phi}$, %
            con $\phi\in(-\pi,\pi]$
        
        \item Anche $\arg\plaquette deve essere anch'es
        %    $$Q \simeq \frac{1}{2\pi}\sum$$
        %\item Nella mia simulazione ho considerato l'approssimazione quenched:
        %    $$S_E = \beta\sum\nolimits_{\plaquette}\left(1-\Re\,\plaquette\right)$$
    \end{itemize}
\end{frame}

\begin{frame}
    \begin{tikzpicture}
        \begin{axis}
            [
                domain = -pi:pi,
                samples = 1000,
                xlabel = $x$,
                xtick = {-3.14159, 0, 3.14159},
                xticklabels = {$-\pi$, $0$, $\pi$},
                legend style = {draw=none},
            ]
            \addplot[fzjblue,thick] {exp(cos(deg(x)))/7.9549};
            \addlegendentry{$\mathcal{N}\,e^{\cos x}$};
            \addplot[color=fzjred,thick] {exp(-.5*x^2)/2.50242};
            \addlegendentry{$\mathcal{N}\,e^{-\frac{x^2}{2}}$};
        \end{axis}
    \end{tikzpicture}
\end{frame}
