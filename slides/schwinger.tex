\part{Modello di Schwinger}
\makepart

\begin{frame}
    \frametitle{Un modello giocattolo}
    \begin{itemize}
        \item Idee e algoritmi possono essere testati su un modello semplificato
        \item Il modello di Schwinger è una teoria $U(1)$ in due dimensioni
        \item Condivide diverse proprietà con la QCD
        \item La principale è il confinamento dei fermioni
        \item $F_{ij}$ è ora il tensore elettromagnetico in 2D
        \item La carica topologica si riduce a:
            $$Q = \frac{g}{4\pi}\int\mathrm d^2x\,\epsilon_{ij}F_{ij}$$
    \end{itemize}
\end{frame}

\begin{frame}
    \frametitle{$Q$ sul reticolo}
    \begin{itemize}
        \item $F_{ij}$ è contenuto nelle plaquette: $\plaquette \simeq e^{iga^2F_{ij}}$
        \item La carica topologica è approssimata da:
            $$Q = \frac{g}{4\pi}\int\mathrm d^2x\,\epsilon_{ij}F_{ij} \simeq %
            \frac{1}{2\pi}\sum\nolimits_\Box\Im\log\plaquette = %
            \frac{1}{2\pi}\sum\nolimits_\Box\arg\plaquette$$
        \item Gli elementi di $U(1)$ sono rappresentati da $e^{i\varphi}$, %
            con $\varphi\in(-\pi,\pi]$
        \item Anche $\arg\plaquette$ deve essere $\in (-\pi,\pi]$
        \item $\arg\plaquette$ si calcola sommando le anomalie dei quattro link, %
            e sostituendo l'angolo ottenuto con l'equivalente $\in (-\pi,\pi]$
            $$\arg\plaquette = \varphi_\Box -\left\lceil\frac{\varphi_\Box-\pi}{2\pi}\right\rceil\,2\pi$$
    \end{itemize}
\end{frame}

\begin{frame}
    \frametitle{Proprietà di $Q$}
    \begin{itemize}
        \item Con condizioni al bordo periodiche $Q \in \mathbb Z$, perché:
            $$\sum\nolimits_\Box\varphi_\Box=0\ \Longrightarrow\ %
            Q=-\sum\nolimits_\Box\left\lceil\frac{\varphi_\Box-\pi}{2\pi}\right\rceil$$
        \item Invertendo l'orientazione dello spazio: $Q \to -Q$
        \item Regioni del reticolo contribuiscono a $Q$ ({\color{fzjblue}istantoni})
        \item Alterando i link interni a un istantone, il suo contributo non cambia
        \item Nel limite al continuo, gli istantoni non si creano né distruggono, %
            possono solo muoversi nello spazio
    \end{itemize}
\end{frame}

\begin{frame}[allowframebreaks]
    \frametitle{Campionamento dei link}
    \begin{itemize}
        \item Nella mia simulazione mi sono limitato all'approssimazione quenched. %
            L'azione è pertanto:
            $$S_E = \beta\sum\nolimits_\Box\left(1-\Re\plaquette\right)$$
        \item Per campionare i link $U$, ho usato un algoritmo locale:
            $$P(U)\mathrm dU \propto e^{\beta\Re(US)}\mathrm dU = e^{\beta k\cos\arg U_0}\mathrm dU_0$$
        \item $S$ è la somma delle due staple connesse a $U$
        \item $k \equiv \left|S\right|$, \ $U_0 \equiv US/k \in U(1)$, \ $\mathrm dU_0 = \mathrm dU$
    \end{itemize}
    \begin{columns}
        \begin{column}{0.4\textwidth}
            \centering
            \begin{itemize}
                \item Bisogna campionare:
                    $$P(\varphi_0) \propto e^{\beta k\cos\varphi_0}$$
                \item È simile a $e^{\beta k\left(1-\frac{\varphi_0^2}{2}\right)}$
                \item Si campiona $\varphi_0$ dalla gaussiana
                \item Si accetta/rigetta con Metropolis-Hastings
            \end{itemize}
        \end{column}
        \begin{column}{0.6\textwidth}
            \centering
            \begin{tikzpicture}
                \begin{axis}
                [
                    width = 0.85\textwidth,
                    height = 0.8\textheight,
                    domain = -pi:pi,
                    samples = 1000,
                    xlabel = $x$,
                    xtick = {-3.14159, 0, 3.14159},
                    xticklabels = {$-\pi$, $0$, $\pi$},
                    legend style = {at={(1.15,1.15)}},
                ]
                    \addplot[fzjblue,thick] {exp(cos(deg(x)))/7.9549};
                    \addlegendentry{$\mathcal{N}\,e^{\cos x}$};
                    \addplot[color=fzjred,thick] {exp(-.5*x^2)/2.50242};
                    \addlegendentry{$\mathcal{N}\,e^{-\frac{x^2}{2}}$};
                \end{axis}
            \end{tikzpicture}
        \end{column}
    \end{columns}
\end{frame}

\begin{frame}
    \frametitle{Misura delle osservabili}
    \begin{itemize}
        \item A ogni spazzata, l'algoritmo locale viene iterato su tutti i link
        \item Le osservabili vengono calcolate dopo ogni spazzata
        \item Gli errori dei valori medi vengono valutati con il binning (o bunching) %
            dividendo le misure in 20 bin
        \item In ogni simulazione sono state iterate 50000 spazzate
        \item Ogni simulazione è partita da una configurazione iniziale calda
        \item Le prime 10000 spazzate sono state scartate per termalizzare
    \end{itemize}
\end{frame}

