\part{Conclusioni}
\makepart

\begin{frame}
    \frametitle{Riassumendo:}
    \begin{itemize}
        \item Sul toro e sulla bottiglia di Klein %
            si ritrovano gli stessi valori per la plaquette e la suscettività topologica
        \item Gli artefatti alla suscettività dovuti alla scatola finita sono trascurabili
            anche per reticoli piccoli
        \item Sulla bottiglia di Klein si evita il problema del freezing
        \item Gli istantoni che danno contributo alla carica totale, al limite al continuo, %
            rimangono inalterati su un toro e cambiano di segno su una bottiglia di Klein
        \item Tuttavia, su varietà non orientabili, $Q$ perde diverse proprietà:
            \begin{itemize}
                \item Non assume più solo valori interi
                \item Non rimane invariata se gli istantoni si spostano
            \end{itemize}
    \end{itemize}
\end{frame}

\begin{frame}
    \frametitle{Una nuova carica}
    \begin{itemize}
        \item È possibile definire una nuova carica che abbia tali proprietà
        \item Si ottiene sommando alla carica 2 volte il loop di Polyakov %
            lungo il bordo di M\"obius
        \item Questa operazione, tuttavia, lascia un grado di libertà di gauge
            $$Q = \frac{1}{2\pi}\sum\nolimits_\Box\left(\arg\plaquette + 2P_0 + 4k\pi\right)$$
        \item Che può essere fissata prendendo il resto modulo 2
            $$Q_\text{eo} = Q \mod 2$$
        \item È accoppiata anche questa con l'assione?
    \end{itemize}
\end{frame}

\begin{frame}
    \frametitle{Ritorna il freezing}
	\begin{columns}
		\begin{column}{0.5\textwidth}
			\begin{tikzpicture}
				\begin{axis}
					[
						width = .95\textwidth,
						height = .95\textwidth,
						enlargelimits = false,
						title = {$\beta = 5.0,\ N = 20$},
						xlabel = Sweeps,
						ylabel = $Q_\text{eo}$,
						ymin = -1, ymax = 2,
                        xtick = {0,500,1000},
                        ytick = {0,1},
					] \addplot+
					[
						mark = none,
						color = fzjblue,
					] gnuplot
					[
						raw gnuplot,
					]
					{
                        plot "../data/cont3EvenOdd.dat" every ::20000::21000 with lines;
					};
				\end{axis}
			\end{tikzpicture}
		\end{column}
		\begin{column}{0.5\textwidth}
			\begin{tikzpicture}
				\begin{axis}
					[
						width = .95\textwidth,
						height = .95\textwidth,
						enlargelimits = false,
						title = {$\beta = 7.2,\ N = 24$},
						xlabel = Sweeps,
                        ylabel = $Q_\text{eo}$,
						ymin = -1, ymax = 2,
                        xtick = {0,500,1000},
                        ytick = {0,1},
					] \addplot+
					[
						mark = none,
						color = fzjblue,
					] gnuplot
					[
						raw gnuplot,
					]
					{
						plot "../data/cont4EvenOdd.dat" every ::20000::21000 with lines;
					};
				\end{axis}
			\end{tikzpicture}
		\end{column}
	\end{columns}
\end{frame}

\begin{frame}
    \begin{center}
        \begin{tikzpicture}
            \begin{axis}
                [ 
                    width = 0.8\textwidth,
                    height = 0.8\textheight,
                    title = Limite continuo,
                    xlabel = $\displaystyle \frac{1}{N^2}$,
                    ylabel = $\displaystyle \left<Q_\text{eo}\right>$,
                    mark size = 1.5pt,
                    xmin = 0, xmax = 0.01,
                    ymin = 0, ymax = 1,
                ]
                \addplot+
                [
                    only marks,
                    mark = square,
                    color = fzjblue,
                    error bars/.cd, y dir = both, y explicit,
                ] table
                [
                    x expr = 1/\thisrowno{3}^2,
                    y expr = \thisrowno{0},
                    y error expr = \thisrowno{1},
                    header = false,
                ] {../data/contEvenOdd.dat};
            \end{axis}
        \end{tikzpicture}
    \end{center}
\end{frame}

\begin{frame}
    \frametitle{Estensione a $SU(3)$}
    \begin{itemize}
        \item $\left<Q^2\right>/N^2$ ha un BIAS in varietà non orientabili
        \item Una definizione più generale di suscettività è:
            $$\chi = \left<\int\mathrm d^4x\,q(x)q(x_0)\right>$$
        \item $q(x)$ è il contributo alla carica di una singola plaquette
        \item $\chi = \left<Q^2\right>/N^2$ con condizioni al bordo periodiche
        \item $\chi \neq \left<Q^2\right>/N^2$ su superfici non orientabili
        \item Il BIAS di $\chi$ diminuisce se $x_0$ si allontana dal bordo di M\"obius%
            \footnote{Come mostrato in ``Lattice QCD on Non-Orientable Manifolds'', \mbox{S. Mages},\mbox{K. Szabo, 2015}}
    \end{itemize}
\end{frame}

